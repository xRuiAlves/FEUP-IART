\documentclass[conference]{IEEEtran}
\IEEEoverridecommandlockouts
% The preceding line is only needed to identify funding in the first footnote. If that is unneeded, please comment it out.
\usepackage{cite}
\usepackage{amsmath,amssymb,amsfonts}
\usepackage{algorithmic}
\usepackage{graphicx}
\usepackage{textcomp}
\usepackage{xcolor}
\usepackage{hyperref}
\usepackage{float}
\def\BibTeX{{\rm B\kern-.05em{\sc i\kern-.025em b}\kern-.08em
    T\kern-.1667em\lower.7ex\hbox{E}\kern-.125emX}}
\begin{document}

\title{Timetabling Problem}

\author{\IEEEauthorblockN{Guilherme Silva (201603647)}
\IEEEauthorblockA{\textit{IART} \\
\textit{FEUP}\\
Porto, Portugal \\
up201603647@fe.up.pt}
\and
\IEEEauthorblockN{Miguel Duarte (201606298)}
\IEEEauthorblockA{\textit{IART} \\
\textit{FEUP}\\
Porto, Portugal \\
up201606298@fe.up.pt}
\and
\IEEEauthorblockN{Rui Alves (201606746)}
\IEEEauthorblockA{\textit{IART} \\
\textit{FEUP}\\
Porto, Portugal \\
up201606746@fe.up.pt}
}

\maketitle

\begin{abstract}
TODO: Abstract
\end{abstract}

\begin{IEEEkeywords}
Artificial Intelligence, Scheduling Problems, Optimization Problems, Optimization algorithms, Hill climbing, Simulated annealing, Genetic Algorithms
\end{IEEEkeywords}

\section{Introduction}

TODO: Introduction

\section{Problem Description}

The \textit{Timetabling Problem} that is subject of study in this project was designd by Ben Paechter, a professor in the Edinburgh University and consists in a reduction of a typical university course timetabling problem.

In this problem, a given set of events has to be scheduled into one hour timeslots, over the course of 5 days with 9 hours of active time each (totaling 45 timeslots). The events take place in a given set of rooms, each with its own size (number of sets), and require certain features (which may or not be available in a room). The students attend a given set of these events. 

The goal is to assign the events to the available rooms, in a such a way that the given \textbf{Hard Constraints} are respected and that the given \textbf{Soft Constraints} are as respected as possible. A given proposed solution is assigned a certain penalty based on the constraints that are not being respected. The optimal solution (if existent for a given input) has a penalty of 0.

\subsection{Hard Constraints}

The hard constraints must be respected by any solution. If any of them are not respected, a penalty of value equal to infinity is assigned to the solution, being them the following:

\begin{itemize}
    \item Only one event can take place in a room at any given timeslot
    \item A student can only attend one event at the same time
    \item The room must be big enough for all the student that are attending the event
    \item The room must satisfy all the features that are required for the event
\end{itemize}

\subsection{Soft Constraints}

The soft constraints are used to evaluate the quality of a valid solution (a solution that respects all the hard constraints), assigning a penalty based on that quality. The assigned penalty is equal to the sum of the penalties of each soft constraint constraint violation, being them the following: 

\begin{itemize}
    \item A student attends an event in the last timeslot of the day. For each student that only attends an event in a given day, a penalty of value 1 is assigned.
    \item A student attends more than two events consecutively. For each consecutively attended event (above 2), a penalty of value 1 is assigned (e.g. 3 consecutive events result in a penalty of 1, 4 consecutive events result in a penalty of 2, an so on). 
    \item A student has a single class on a day. For each student that attends only one event in a given day, a penalty of 1 is assigned.
\end{itemize}

\section{Problem Formulation}

TODO: Problem Formulation

\section{Related Work}

TODO: Related Work

\section{Conclusions and Future Work}

TODO: Conclusions and Future Work

\begin{thebibliography}{00}
    
\bibitem{b1} "Unblock me FREE.". Google Play. February 28, 2019.\href{https://play.google.com/store/apps/details?id=com.kiragames.unblockmefree}{https://play.google.com/store/apps/details?id=com.kiragames.unblo\\ckmefree}.
\bibitem{b2} Fogleman, Michael. "Solving Rush Hour, the Puzzle.". July, 2018. \href{https://www.michaelfogleman.com/rush/}{https://www.michaelfogleman.com/rush/}.
\bibitem{b3} "Bitboard.". Wikipedia - The free Encyclopedia. December 6, 2018. \href{https://en.wikipedia.org/wiki/Bitboard}{https://en.wikipedia.org/wiki/Bitboard}.
\bibitem{b4} Littman, Michael. "Programming Assignment P1 - What A* Rush". Priceton. 2012. \href{https://www.cs.princeton.edu/courses/archive/fall12/cos402/assignments/programs/rushhour/}{https://www.cs.princeton.edu/courses/archive/fall12/cos402/assignments\\/programs/rushhour/}.
\bibitem{b5} Findling, Rainhard. "The RushHour Puzzle – an Artificial Intelligence Toy Problem.". April 4, 2012. \href{http://geekoverdose2.rssing.com/browser.php?indx=39804402\&item=1}{http://geekoverdose2.rssing.com/browser.php?indx=39804402\&item=1}.

\end{thebibliography}

\end{document}
